\documentclass[12pt]{article}
\usepackage[margin=1in]{geometry}
\geometry{letterpaper}
\usepackage{graphicx}
\usepackage{setspace}
\usepackage{amssymb}
\usepackage{amsmath}
\usepackage{epstopdf}
\usepackage{natbib}

\title{Summary of literature for ``How to be happy when your data are SAD''}

\author{}


\begin{document}
\maketitle

\paragraph{\cite{williamson2005}}
Central limit theorem not for SAD (unless neutral populations) and
veil line doesn't exist.  Should check out other papers for discussion
of sampling like Dewdney (1998).  Check out Dewdney in genreal.


\paragraph{\cite{dewdney1998}}
There is no veil line, species are sampled according to a
hypergeometric which can be approximated by a Poisson.

\paragraph{\cite{dewdney2000}}
Too much rarity for log-normal to be valid. Shits on negbinom, but
negbinom can produce the right amount of rarity

\paragraph{\cite{chisholm2007}}
Shows that veil line is valid on log scale but that has no relevance
for actually fitting data.  Shows more robustly than
\citep{dewdney1998} that the shape of the SAD is impervious to
sampling, but only when total community is large relative to sample, a
condition that is likely to hold.

\paragraph{\cite{dewdney2003}}
All species ``orbit'' the mean of the community and have approximately
equal birth and death rates (i.e. birth $\approx$ death, not
neccesarily neutrality) ``stochastic communities hypothesis'' proposed
as the overall mechanism governing abundances in all natural
communities. It is not a single mechanism, per se, but the net effect
of all environmental influences.

Hughes, R.G. (1986) Theories and models of species abundance. American Naturalist, 128, 879–899.



Dewdney, A.K. (2003) The stochastic community and the logistic-J distribution. Acta Oecologia, 24, 221–229.

\bibliographystyle{ecol_let}
\bibliography{sad.bib}

\end{document}