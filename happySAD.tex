\documentclass[12pt]{article}
\usepackage[margin=1in]{geometry}
\geometry{letterpaper}
\usepackage{graphicx}
\usepackage{setspace}
\usepackage{amssymb}
\usepackage{amsmath}
\usepackage{epstopdf}

\title{How to be happy when your data are SAD}

\author{Andrew J. Rominger, John Harte}


\begin{document}
\maketitle

The species abundance distribution (SAD) is a fundamental pattern of
biodiversity; any theory of biodiversity, in addition to whatever
other patterns it may predict, should also make meaningful and
accurate predictions of the SAD. While the SAD is likely insufficient
to differentiate mechanistic hypotheses of community assembly its
universality begs prediction and indeed can be used to reject some,
but not all, theories of biodiversity. Reviews of theories predicting
SADs have been presented ad nauseam. Here we sketch an outline of
these theories for the sake of tradition but go one step further and
use arguments from information theory, probability theory, simulation
and empirical data to make explicit judgment about which should be
abandoned and which warrant further consideration and
investigation. We also use an extensive simulation experiment to
investigate the most robust means for analyzing species abundance
data. Information theoretic methods for comparing models of the SAD
are the most robust means available; this needs no validation by
simulation. Using these methods we find strongest support for the
negative binomial distribution. To directly evaluate the fit of one
SAD model to data we find that a z-score based on the log likelihood
is the most efficient method (i.e. $(logLik_{obs} -
E[logLik_{theoretical}]) / SD[logLik_{theoretical}]$ where $logLik_{obs}$ is
the observed log likelihood and $logLik_{theoretical}$ refers to the
theoretical SAD calculated analytically or by
simulation). Additionally, random subsampling SAD data has no
meaningful effect on the identification of the most supported model
given the data. Conversely, binning species abundance data hugely
distorts any ability to correctly estimate models.  Taking these
results together we find that the log-normal distribution with
associated veil line argument is invalid and the recently proposed
gambin model can never be realistically estimated from data; both
should be abandoned.  The negative binomial has a strong tradition in
ecology and should be re-invigorated.  Using read data and the sketch
of a probabilistic, non-neutral community assembly model, we conclude
by exploring the causes and consequences of the wide support for the
negative binomial.


\section{On the utility of SADs}
SADs are a fundamental pattern of biodviersity, any thoery of
biodiversity, in addition to whatever other patterns it may predict,
should also make meaningful and accurate predictions of the SAD

\section{SAD theories}
Reviews of theories predicting SADs have been presented ad
nauseum. Here we sketch an outline of these theories for the sake of
tradition but go one step further and make explicit judgement about
which should be abandonded and which warrent further consideration and
investigation.

\subsection{A SAD history}
Broken stick, log series, log normal, neutral theory, mechanismless
theory

\subsection{Bad SAD theories}
The log-normal is not valid and not useful, nor is it well supported
by data. The veil-line is a misconception from poorly thought-out
subsampling as discussed in Section \ref{sec:samp}.

The Gambin model is also not valid nor well supported, its support
arrises from innapropriate data handling as discussed in Section
\ref{sec:statPrac}. Mathematically it is also a basterdization of the
much more correct and informative negative binomial model of species
abundance.

\subsection{SAD theories for the future}
Return to Fisher's formulation and Preston's ergodic hypothesis


\section{SAD subsamples}
\label{sec:samp}

SADs are sampled by indiviudal not by species. The veil line arrises
by inappropriately sampling by species.

SADs have a discrete spatio-temporal extent. This is arbitrary but
important: The log normal shape arrises by mixing samples
(i.e. multiple years of moth trapping, combining distinct spatial
samples)

How SADs scale is another issue and an interesting topic on inquery.
The fact that there is scale depenence in SADs doesn't negate the fact
that sampling them must be viewed as discrete in space, time and
taxonomy.

\section{Best and worst statistical practices for comparing
  theoretical and emperical SADs}
\label{sec:statPrac}

Do not bin data. Deal with subsampling appropriately

\subsection{Comparing theory to data}
Likelihood approaches, RAD, a note on hierarchical models and how
commonly used normal hyperdistribuitons might nearly always be wrong.

\section{Biological and statistical limits to interpreting SADs}

How small a sample is too small?

Can we distinguish log normal hyperdistributions from Gamma
hyperdistributions?

Recent mechanismless theory says SADs might just be statistical
outcome, so how much should we really infer from them?

\end{document}