\documentclass[12pt]{article}
\usepackage[margin=1in]{geometry}
\geometry{letterpaper}
\usepackage{graphicx}
\usepackage{setspace}
\usepackage{amssymb}
\usepackage{amsmath}
\usepackage{epstopdf}
\usepackage{natbib}

\title{How to be happy when your data are SAD}

\author{Andrew J. Rominger, John Harte}


\begin{document}
\maketitle

\clearpage

The species abundance distribution (SAD) is a fundamental pattern of
biodiversity; any theory of biodiversity, in addition to whatever
other patterns it may predict, should also make meaningful and
accurate predictions of the SAD. While the SAD is likely insufficient
to differentiate mechanistic hypotheses of community assembly its
universality begs prediction and indeed can be used to reject some,
but not all, theories of biodiversity. Reviews of theories predicting
SADs have been presented ad nauseam. Here we sketch an outline of
these theories for the sake of tradition but go one step further and
use arguments from information theory, probability theory, simulation
and empirical data to make explicit judgment about which should be
abandoned and which warrant further consideration and
investigation. We also use an extensive simulation experiment to
investigate the most robust means for analyzing species abundance
data. Information theoretic methods for comparing models of the SAD
are the most robust means available; this needs no validation by
simulation. Using these methods we find strongest support for the
negative binomial distribution. To directly evaluate the fit of one
SAD model to data we find that a z-score based on the log likelihood
is the most efficient method (i.e. $(logLik_{obs} -
E[logLik_{theoretical}]) / SD[logLik_{theoretical}]$ where $logLik_{obs}$ is
the observed log likelihood and $logLik_{theoretical}$ refers to the
theoretical SAD calculated analytically or by
simulation). Additionally, random subsampling SAD data has no
meaningful effect on the identification of the most supported model
given the data. Conversely, binning species abundance data hugely
distorts any ability to correctly estimate models.  Taking these
results together we find that the log-normal distribution with
associated veil line argument is invalid and the recently proposed
gambin model can never be realistically estimated from data; both
should be abandoned.  The negative binomial has a strong tradition in
ecology and should be re-invigorated.  Using read data and the sketch
of a probabilistic, non-neutral community assembly model, we conclude
by exploring the causes and consequences of the wide support for the
negative binomial.

\clearpage

\section{Introduction}

The distribution of abundances across species has remained a central
pattern in ecology for more than 100 years
\citep{RaunkaierLawFrequence}. The modern importance of species
abundance distributions (SADs) began with theoretical attempts by
\cite{fisher} and \cite{preston} to derive the expected number of
sampled species with a given abundance. This work has culminated in a
current diversity of theories predicting specific shapes of
SADs. Indeed the utility of the SAD is that it captures a wealth of
information about an ecological community into a single numerical
summary that can be readibly compared to quantitative, mathematical
predictions based on the hypothesized mechanisms underlying community
assembly and maintenance.  Commonness and rarity are also of inhernent
interest in their own right due to the importance of population size,
which must scale with abundance, to the genetics, evolution and
conservation of species. As such, any useful thoery of biodiversity,
in addition to whatever other patterns it may predict, should make
meaningful and accurate predictions of the SAD. Testing these
predictions with real data is of equal importance to developing the
predictions themselves.  Both theory development and testing have seen
success \citep{} but are also fraught with poor practices. Here I 1)
briefly summarize \citep[not needing to re-invent more detailed
reviews][]{stuff} the most relevant SAD theories, laying to rest
several irrelevant theories; 2) present a set of best statistical
practices for SAD testing, clarifying misconceptions such as the
veil-line; and 3) explore the future of SAD theories, including the
exciting universality and utility of the negative binomial
distribution.

\subsection{A SAD history}
Reviews of theories predicting SADs have been presented ad
nauseum. Here we sketch an outline of these theories for the sake of
tradition but go one step further and make explicit judgement about
which should be abandonded and which warrent further consideration and
investigation.
% Broken stick, log series, log normal, neutral theory, mechanismless
% theory

\subsection{Bad SAD theories}
The log-normal is not valid and not useful, nor is it well supported
by data. The veil-line is a misconception from poorly thought-out
subsampling as discussed in Section \ref{sec:samp}.

The Gambin model is also not valid nor well supported, its support
arrises from innapropriate data handling as discussed in Section
\ref{sec:statPrac}. Mathematically it is also a basterdization of the
much more correct and informative negative binomial model of species
abundance.

\subsection{SAD theories for the future}
Return to Fisher's formulation and Preston's ergodic hypothesis


\section{SAD subsamples and mixtures}
\label{sec:samp}

SADs are sampled by indiviudal not by species. The veil line arrises
by inappropriately sampling by species.

SADs have a discrete spatio-temporal extent. This is arbitrary but
important: The log normal shape arrises by mixing samples
(i.e. multiple years of moth trapping, combining distinct spatial
samples)

How SADs scale is another issue and an interesting topic on inquery.
The fact that there is scale depenence in SADs doesn't negate the fact
that sampling them must be viewed as discrete in space, time and
taxonomy.

Mixing SADs leads to log normality

\section{Best and worst statistical practices for comparing
  theoretical and emperical SADs}
\label{sec:statPrac}

Do not bin data. Deal with subsampling appropriately

\subsection{Comparing theory to data}
Likelihood approaches, RAD, a note on hierarchical models and how
commonly used normal hyperdistribuitons with log and logit link
functions might nearly always be wrong.

\section{Biological and statistical limits to interpreting SADs}

How small a sample is too small?

Can we distinguish log normal hyperdistributions from Gamma
hyperdistributions?

Recent mechanismless theory says SADs might just be statistical
outcome, so how much should we really infer from them?

\end{document}